% Mike Curry, Kareem Mohamed and Bradford Smith
% CS 370 Group Project 2 SPOJ LCMSUM slides.tex
% 05/02/2016
% "We pledge our honor that we have abided by the Stevens Honor System."

\documentclass{beamer}
\mode<presentation>
{
  \usetheme{default}
  \usecolortheme{default}
  \usefonttheme{default}
  \setbeamertemplate{navigation symbols}{}
  \setbeamertemplate{caption}[numbered]
  \setbeamertemplate{footline}[frame number]
}

\usepackage[english]{babel}
\usepackage[utf8x]{inputenc}
\usepackage{listings}

\title[LCMSUM]{LCM Sum (LCMSUM)}
\subtitle{http://www.spoj.com/problems/LCMSUM}
\author{Mike Curry \& Kareem Mohamed \& Bradford Smith}
\institute{CS 370}
\date{May 2, 2016}

\begin{document}

\begin{frame}
  \titlepage
\end{frame}

\begin{frame}{Outline}
  \tableofcontents
\end{frame}

\section{Problem Statement}

\begin{frame}{Problem Statement}
  \begin{itemize}
  \item Given some n, calculate the sum:
    \begin{equation*}
      \boxed{LCM(1, n) + LCM(2, n) + \cdots + LCM(n, n)}
    \end{equation*} \vspace{10mm}
  \item For example, given $n = 3$:
    \begin{align*}
      LCM(1, 3) + LCM(2, 3) + LCM(3, 3) &= \\
      &= 3 + 6 + 3 \\
        &= 12
    \end{align*}
  \end{itemize}
\end{frame}

\section{Approaching a Solution}

\subsection{Failed Attempts}

\begin{frame}{Computing LCM using GCD}
  \begin{itemize}
  \item The LCM can be represented in terms of the GCD
    \begin{equation*}
      \boxed{LCM(a, b) = \frac{|a * b|}{GCD(a, b)}}
    \end{equation*}
  \end{itemize}
\end{frame}

\begin{frame}[fragile]{First GCD Attempt}
  \begin{itemize}
  \item GCD can be computed as
    \begin{equation*}
      \boxed{GCD(a, b) = (a+b) - (a*b) + 2*\sum_{k = 1}^{a - 1}{\left\lfloor\frac{k*b}{a}\right\rfloor}}
    \end{equation*}
    \begin{lstlisting}[language=C, basicstyle=\footnotesize\ttfamily]
unsigned long gcd(int c, int d)
{
      int k;
      unsigned long sum = 0;
      for (k = 1; k <= (c-1); ++k)
      {
          sum += (k*d)/c;
      }
      return (unsigned long) (c+d)-(c*d)+(2*sum);
}
    \end{lstlisting}
  \end{itemize}
\end{frame}

\begin{frame}[fragile]{Second GCD Attempt}
  \begin{itemize}
  \item Using Euclid's Algorithm
    \begin{lstlisting}[language=C, basicstyle=\footnotesize\ttfamily]
unsigned long gcd(unsigned long c, unsigned long d)
{
    unsigned long r;

    if((c == 0) || (d == 0))
        return 0;
    while(1)
    {
        r = c % d;
        if(r == 0)
            break;
        c = d;
        d = r;
    }
    return d;
}
    \end{lstlisting}
  \end{itemize}
\end{frame}

\begin{frame}[fragile]{Third GCD Attempt}
  \begin{itemize}
  \item Using Euler Totient Function
    \begin{equation*}
      \boxed{GCD(a, b) = \sum_{k|a \wedge k|b}{\varphi(k)}}
    \end{equation*}
    \begin{lstlisting}[language=C, basicstyle=\footnotesize\ttfamily]
unsigned long gcd_phi(unsigned long c, unsigned long d)
{
    unsigned long ans = 0;
    for (unsigned long k = 1; k * k <= d; ++k)
    {
        if (c % k == 0 && d % k == 0)
        {
            ans += phi[k];
        }
    }
    return ans;
}
    \end{lstlisting}
  \end{itemize}
\end{frame}

\subsection{Conclusions}

\begin{frame}{Our Findings}
  \begin{itemize}
  \item Computing each LCM was too slow \vskip 10mm
  \item Mathematica/SymPy were unable simplify both formulas. \vskip 10mm
  \item We found a lot of papers written on a GCD-Sum function but they were
    too dense to understand in our timespan.
  \end{itemize}
\end{frame}

\end{document}
